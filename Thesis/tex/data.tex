\section{Data}
The raw LiDAR point cloud data of this area were retrieved from Publieke Dienstverlening op de Kaart (PDOK), a geo-information service of the Dutch government~\citep{PDOK2016AHN3}. The data belongs to the Actueel Hoogtebestand Nederland 3 (AHN3) dataset, which is being scanned from 2014 to 2019 and currently covers about 45\% of the country~\citep{AHN2016inwinjaren, PDOK2016AHN3}. The dataset has a point density of 6 to 10 points per square meter and has information on the height with multiple discrete return values and includes intensity data. The dataset is retrieved in the first quarter of each year (our study area was scanned in 2015) and consequently the deciduous vegetation was leafless during this time~\citep{AHN2016inwinjaren}. While this decreases the reflectance of vegetation, a visual inspection showed the return signal is still strong enough to retrieve a good scan of the vegetation.