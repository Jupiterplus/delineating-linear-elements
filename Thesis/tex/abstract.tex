\begin{abstract}
	Intensification of farming and the associated increased size of agricultural fields has caused a major decrease in linear vegetation elements such as tree lines and hedges in rural areas across Europe. These elements are considered being a part of the cultural landscape and functions as corridors in the local ecosystem, hosting many animal and plant species. An inventory of the spatial distribution of these elements is therefore valuable for the conservation of habitats and promoting biodiversity. 
	
	Our aim was to develop an automated method to detect these elements using data available nationwide. LiDAR point clouds are increasingly available at these national scales and the 3D view makes them very suitable for detecting vegetation. Geometric properties of the LiDAR points and their neighbourhoods were calculated. Based on the planarity and spherecity properties most of the irrelevant points were removed. To separate the vegetation from the remaining points a random forest classifier was used. A further subdivision into low vegetation and trees was based on the height differences around each return. The vegetation classes were segmented using a region growing method based on a newly developed rectangularity approach. The resulting rectangular regions were then assessed for their linearity by analysing their elongatedness.
	
	The relevant vegetation was segregated well and although small improvements can still be made, those would not result in a better delineation of linear vegetation elements, since the misclassified points are dispersed. Although extraction of irregular vegetation patterns can be optimized, the newly developed method proved to successfully separate the majority of linear and non-linear vegetation elements in the study area, a Dutch agricultural landscape.
	
	Utilizing a high performance computing paradigm the developed method can be used over large areas. The results can be used as an extra indicator in ecosystem and biodiversity assessments. Additionally it can be used in ecological research, for example to refine spatial distribution models of flora and fauna. Furthermore, the method has potential to also extract other elongated objects from point clouds, such as roads and ditches.
\end{abstract}